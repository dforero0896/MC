\documentclass{article}
\usepackage[utf8]{inputenc}
\usepackage[spanish]{babel}
\setlength{\parindent}{0cm}
\title{Primer Documento Proyecto Final}
\author{Daniel Forero}
\date{\today}
\usepackage{graphicx}
\usepackage{anysize}
\marginsize{3cm}{2.5cm}{2.5cm}{3cm}
\begin{document}
\maketitle
\section{Objetivos}
\subsection{Principales}
\begin{enumerate}
\item Modelar paquetes de onda que respondan a parámetros reales encontrados en las funciones de onde de una partícula.
\item Visualizar el comportamiento de dichos paquetes de onda en distintas situaciones o bajo distintos potenciales. Haciendo necesaria la solución de la ecuación de Schrödinger.
\item Visualizar el comportamiento de éstos al colisionar con otro similar, o un obstáculo.
\item Extraer información del sistema tal como masa, momento lineal, funciones de probabilidad, etc. para distintos momentos.
\end{enumerate}

\subsection{Secundarios}
\begin{enumerate}
	\item Hacer uso de diversas herramientas computacionales en el análisis de las propiedades de los paquetes de ondas tal y como lo son solución numérica de ecuaciones diferenciales y transformadas de Fourier.
	\item Hacer uso de la transformada ondícula (\textit{wavelet}) y comparar el resultado con la información obtenida de la transformada de Fourier.
\end{enumerate}


\section{Marco Teórico}
La dualidad onda-partícula de la materia permite analizar sistemas físicos cuánticos a partir del uso de paquetes de ondas a manera de partículas. Dichos paquetes de ondas deben contener toda la información de la partícula y permite, además obtener una función de probabilidad para la posición de la partícula en un cierto instante.\\
En este orden de ideas, es posible ver, a partir de éstos paquetes, como se comportaría una partícula en situaciones como lo son, la acción de un potencial, el choque con otra partícula y el choque con un obstáculo. Permitiendo ver cambios en la energía, la masa, la velocidad y demás información que el paquete pueda contener.\\
En algunas de estas situaciones es necesaria la solución de la ecuación de Schrödinger, definida como:
\begin{equation}
-\frac{\hbar^2}{2m}\nabla^2 \Psi + U(x)\Psi=-\i \hbar \frac{\partial \Psi}{\partial t}
\end{equation}

Donde $\Psi$ es la función de onda en el tiempo y el espacio.\\
En lo posible se utilizarán las rutinas disponibles para Python en la solución de ecuaciones y la aplicación de transformadas.
\end{document}